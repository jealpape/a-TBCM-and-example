\documentclass[a4paper,10pt]{article}
\usepackage[utf8]{inputenc}
\usepackage{amsmath,amsfonts,amsthm}

%opening
\title{Triangular Body-Cover Model: \\ Description and Main Equations}
\author{Gabriel A. Alzamendi}
\date{\today}

\begin{document}

\maketitle

\begin{abstract}
 This document describes the basics concepts in the Triangular Body-Cover model (TBCM) 
 and the most important equations involved, as introduced in \cite{galindo_modeling_2017}. 
 This document pays special attention to some improved/revisited rules.
\end{abstract}

\section{Dynamic equations in the TBCM}
TBCM is a lumped-element biomechanical model of the vocal folds introduced in  \cite{galindo_modeling_2017}. 
It is composed of three masses representing the cover-body structure of the human vocal folds, 
where upper $m_{u}$ and lower $m_{l}$ masses constitute the cover layer,
and the body mass $m_{b}$ emulates the internal body in the vocal fold. 
The dynamic of the TBCM is described through a second order differential equations:
\begin{equation} \label{Eq_S1s0_01}
  \begin{aligned}
    m_{u} \, \ddot{x}_{u} &= F_{k\,u} + F_{d\,u} - F_{kc} + F_{e\,u} + F_{col\,u}, \\
    m_{l} \, \ddot{x}_{l} &= F_{k\,l} + F_{d\,l} + F_{kc} + F_{e\,l} + F_{col\,l}, \\
    m_{b} \, \ddot{x}_{b} &= F_{k\,b} + F_{d\,b} - (F_{k\,u} + F_{d\,u} + F_{k\,l} + F_{d\,l}),
  \end{aligned}
\end{equation}
where for $* \in \{u,l,b\}$ the variables $x_{*}$ describe the displacement of the masses with respect to 
the rest (equilibrium) position $x_{0\,*}$.
The force terms correspond to the (elastic) spring response ($F_{k\,*}$ and $F_{kc}$), the damping ($F_{d\,*}$), 
the reaction during left-right vocal folds collision ($F_{col\,*}$), and aerodynamics force due to 
glottal airflow ($F_{e\,*}$). The masses parameters $(m_{u}, \, m_{l}, \, m_{b} )$ are obtained as a function
of the activation of laryngeal muscles \cite{titze_rules_2002}.

According to the TBCM, the phonatory (triangular) glottal configuration is a function of arytenoid cartilage posturing. 
Let $\xi_{u}$ and $\xi_{l}$ be the position of the posterior ends of the upper and lower masses, 
respectively, due to medial-lateral vocal process displacement. 
The equilibrium positions for the upper and lower masses are equal to the average horizontal glottal position:
\begin{equation} \label{Eq_S1s0_02}
  x_{0\,u} = \frac{\xi_{u}}{2}, \qquad \text{and} \qquad
  x_{0\,l} = \frac{\xi_{l}}{2}.
\end{equation}
For the body mass, the equilibrium position $x_{0\,b} = 0.3$ cm.

\subsection{Elastic spring forces}
The spring forces are computed as follows:
\begin{equation} \label{Eq_S1s1_01}
  \begin{aligned}
    F_{k\,u} & = -k_{u} \, \left[ \left[ (\delta_{u} - x_{0\,u}) - (x_{b} - x_{0\,b}) \right] +
                \eta_{u} \left[ (\delta_{u} - x_{0\,u}) - (x_{b} - x_{0\,b}) \right]^3 \right], \\
    F_{k\,l} & = -k_{l} \, \left[ \left[ (\delta_{l} - x_{0\,l}) - (x_{b} - x_{0\,b}) \right] +
                \eta_{l} \left[ (\delta_{l} - x_{0\,l}) - (x_{b} - x_{0\,b}) \right]^3 \right], \\
    F_{k\,b} & = -k_{b} \, \left[ \left(x_{b} - x_{0\,b} \right) +
                \eta_{b} \left(x_{b} - x_{0\,b} \right)^3 \right], \\
    F_{kc} & = -k_{c} \, \left[ (\delta_{u} - x_{0\,u}) - (\delta_{l} - x_{0\,l}) \right],
  \end{aligned}
\end{equation}
where $\delta_{u} = x_{u} + \frac{\xi_{u}}{2}$ and $\delta_{l} = x_{l} + \frac{\xi_{l}}{2}$.

The expression can be written as:
\begin{equation} \label{Eq_S1s1_02}
  \begin{aligned}
    F_{k\,u} & = -k_{u} \, \left[ \left( x_{u} - x_{b} \right) +
                \eta_{u} \left( x_{u} - x_{b} \right)^3 \right], \\
    F_{k\,l} & = -k_{l} \, \left[ \left( x_{l} - x_{b} \right) +
                \eta_{l} \left( x_{l} - x_{b} \right)^3 \right], \\
    F_{k\,b} & = -k_{b} \, \left[ \left(x_{b} - x_{0\,b} \right) +
                \eta_{b} \left(x_{b} - x_{0\,b} \right)^3 \right], \\
    F_{kc} & = -k_{c} \, \left[ x_{u} - x_{l} \right].
  \end{aligned}
\end{equation}
The elastic parameters $(k_{u}, \, k_{l}, \, k_{b}, \, k_{c})$ are obtained as a function
of the activation of laryngeal muscles \cite{titze_rules_2002}.

\subsection{Damping forces}
Damping forces are \cite{galindo_modeling_2017,lucero_simulations_2005}:
\begin{equation} \label{Eq_S1s2_01}
  \begin{aligned}
    F_{d\,u} & = -2\zeta_{u}\sqrt{m_{u} k_{u}} \left( 1 + 150 |\delta_{u}| \right)
                        \left( \dot{x}_{u} - \dot{x}_{b} \right), \\
    F_{d\,l} & = -2\zeta_{l}\sqrt{m_{l} k_{l}}  \left( 1 + 150 |\delta_{l}| \right)
                        \left( \dot{x}_{l} - \dot{x}_{b} \right), \\
    F_{d\,b} & = -2\zeta_{b}\sqrt{m_{b} k_{b}} \left( \dot{x}_{b} \right).
  \end{aligned}
\end{equation}
where $\delta_{u} = x_{u} + \frac{\xi_{u}}{2}$ and $\delta_{l} = x_{l} + \frac{\xi_{l}}{2}$.

\subsection{Collision forces}
In order to compute the reaction forces due to the collision of left/right vocal folds,
glottis membranous area and contact area during collision are first described. 

According to the TBCM, the parametric representations of the rest configuration for the cover masses are
$r_{*}(z) = \xi_{*} \left(1 - \frac{z}{L_{g}} \right)$, $0\leq z \leq L_{g}$, and $* \in \{u,l\}$.
Then, the phonatory configuration can be computed as follows:
\begin{equation} \label{Eq_S1s3_01}
  w_{*}(z) = x_{*} + \xi_{*} \left(1 - \frac{z}{L_{g}} \right).
\end{equation}
Thus, vocal fold collision occurs for all $z$ given that $w_{*}(z) \leq 0$.

\subsubsection{Region under collision}
In the vertical axis, the region under collision is delimited by two extreme points.
Solving for $w_{*}(z) = 0$, one of these extreme points is computed:
\begin{equation} \label{Eq_S1s3_02}
  \tilde{z} = \min\left(0, \, \max\left(\frac{ x_{*} + \xi_{*} }{\xi_{*}}, \, 1\right) \right) \, L_{g}.
\end{equation}
The other extreme points depends on $\xi_{*}$. For the case $\xi_{*}>0$, the vertical region under
collision is $[\tilde{z},L_{g}]$, whereas if $\xi_{*}<0$ the region is $[0,\tilde{z}]$.
In the case $\xi_{*}=0$, the TCBM tends to the original \emph{body-cover} model \cite{story_voice_1995}.

Let $\alpha_{*}$ be the portion of length of the cover masses under collision, 
with $0 \leq \alpha_{*} \leq 1$ and $* \in \{u,l\}$. Following the definition of the collision region
introduced above, the formulation of $\alpha_{*}$ depends on $x_{*}$ and $\xi_{*}$:
\begin{description}
  \item[Case $\xi_{*}=0$:] parallel bars as in the body-cover model,
    \begin{equation}
      \alpha_{*} = \frac{1 - \operatorname{sgn}(x_{*} )}{2}.
    \end{equation}
  \item[Case $\xi_{*}>0$:] triangular configuration with the lateral displacement in the posterior end,
    \begin{equation}
      \alpha_{*} = \min\left( \max\left(0, \, -\frac{x_{*} }{\xi_{*}} \right), \, 1 \right).
    \end{equation}
  \item[Case $\xi_{*}<0$:] triangular configuration with the lateral displacement in the anterior end,
    \begin{equation}
      \alpha_{*} = \min\left( \max\left(0, \, \frac{x_{*}+\xi_{*}  }{\xi_{*}} \right), \, 1 \right).
    \end{equation}
\end{description}

\subsubsection{Rule for collision forces}
Considering the definitions introduced in the previous section, 
the resulting collision forces $F_{col\,*}$, $* \in \{u,l\}$, along the region under collision is:
\begin{equation} \label{Eq_S1s3_04}
  F_{col\,*} = \frac{1}{L_{g}} \int_{z_1}^{z_2} -\hat{h}_{col\,*} \left[
                    w_{*}(z) + \hat{\eta}_{col\,*} w_{*}(z)^3 \right] \, dz,
\end{equation}
where $z_1$ and $z_2$ are the inferior and superior extremes of the collision region.
Applying integration by substitution, Eq. \eqref{Eq_S1s3_04} is integrated out:
\begin{equation} \label{Eq_S1s3_05}
  F_{col\,*} = \frac{\hat{h}_{col\,*}}{\xi_{*}} \left[\frac{1}{2} w_{*}(z)^2 +
                      \frac{\hat{\eta}_{col\,*}}{4} w_{*}(z)^4  \right]_{z_1}^{z_2}.
\end{equation}

Next, the different solutions are obtained:
\begin{description}
  \item[Case $\xi_{*} \geq 0$:]
    Correspond to a triangular configuration with the lateral displacement in the posterior end.
    For the case $x_{*} \leq 0$, the inferior and superior extremes are $z_1 = \tilde{z} $ and $z_2 = L_{g}$,
    and $w_{*}(z_1) = 0$ and $w_{*}(z_2) = x_{*}$.
    The resulting collision forces is:
    \begin{equation}
      \begin{aligned}
        F_{col\,*} & = \frac{\hat{h}_{col\,*}}{\xi_{*}} \left[\frac{1}{2} x_{*}^2 +
                         \frac{\hat{\eta}_{col\,*}}{4} x_{*}^4  \right]
                     = - \alpha_{*} \hat{h}_{col\,*} \left[\frac{1}{2} x_{*} +
                         \frac{\hat{\eta}_{col\,*}}{4} x_{*}^3  \right]     \\
                   & = - \alpha_{*} \hat{h}_{col\,*} \left\{
                            \left( \delta_{*} - (1 - \alpha_{*}) \frac{\xi_{*}}{2} \right)
                        + \hat{\eta}_{col\,*} \left( \delta_{*} - (1 - \alpha_{*}) \frac{\xi_{*}}{2} \right) 
                        ^{\phantom{2}} \right. \\
                   & \hspace{3cm}   \left.  \left[
                            \left( \delta_{*} - (1 - \alpha_{*}) \frac{\xi_{*}}{2} \right)^2 + 
                            \left( \alpha_{*} \frac{\xi_{*}}{2} \right)^2  \right] \right\},
      \end{aligned}
    \end{equation}
    where $\delta_{*} = x_{*} + \frac{\xi_{*}}{2}$.

  \item[Case $\xi_{*} < 0$:]
    Correspond to a triangular configuration with the lateral displacement in the anterior end.
    For the case $x_{*} > 0$, the inferior and superior extremes are $z_1 = 0 $ and $z_2 = \tilde{z}$,
    and $w_{*}(z_1) = x_{*}+\xi_{*}$ and $w_{*}(z_2) = 0$.
    The resulting collision forces is:
    
    \begin{equation}
      \begin{aligned}
        F_{col\,*} & = \frac{\hat{h}_{col\,*}}{\xi_{*}} \left[\frac{-1}{2} (x_{*}+\xi_{*})^2 -
                         \frac{\hat{\eta}_{col\,*}}{4} (x_{*}+\xi_{*})^4  \right]     \\
                   & = - \alpha_{*} \hat{h}_{col\,*} \left[\frac{1}{2} (x_{*}+\xi_{*}) +
                         \frac{\hat{\eta}_{col\,*}}{4} (x_{*}+\xi_{*})^3  \right]     \\
                   & = - \alpha_{*} \hat{h}_{col\,*} \left\{
                            \left( \delta_{*} - (1 - \alpha_{*}) \frac{(-\xi_{*})}{2} \right)
                        + \hat{\eta}_{col\,*} \left( \delta_{*} - (1 - \alpha_{*}) \frac{(-\xi_{*})}{2} \right)
                        ^{\phantom{2}} \right. \\
                   & \hspace{3cm}   \left.  \left[
                            \left( \delta_{*} - (1 - \alpha_{*}) \frac{(-\xi_{*})}{2} \right)^2 + 
                            \left( \alpha_{*} \frac{\xi_{*}}{2} \right)^2  \right] \right\},
      \end{aligned}
    \end{equation}
    where $\delta_{*} = x_{*} + \frac{\xi_{*}}{2}$.
\end{description}


\section{Membranous glottal area}
The membranous glottal area $A_m$ is delimited for the triangular posture of the glottal folds, 
from the posterior vocal processes to the anterior ends. This parameter can be computed as follows:
\begin{equation} \label{Eq_S2s0_01}
  A_m = 2 \begin{cases}
            (1 - \alpha_{*}) L_{g} \left(\delta_{*} + \alpha_{*} \frac{\xi_{*}}{2} \right),
                & 0 \leq \xi_{*}, \\
            (1 - \alpha_{*}) L_{g} \left(\delta_{*} + \alpha_{*} \frac{(-\xi_{*})}{2} \right),
                & \xi_{*} <0,
          \end{cases}
\end{equation}
where $\delta_{*} = x_{*} + \frac{\xi_{*}}{2}$.
Taking into account the definition of $\alpha_{*}$, membranous area formulation for the original
body-cover model is obtained through Eq. \eqref{Eq_S2s0_01} in the case $\xi_{*}=0$.


\newpage

% Bibliography
\bibliographystyle{abbrv} 
\bibliography{references}



\end{document}
